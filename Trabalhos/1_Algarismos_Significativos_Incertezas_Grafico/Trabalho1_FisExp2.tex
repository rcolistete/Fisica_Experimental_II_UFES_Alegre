\documentclass[a4paper,11pt]{report}

\oddsidemargin -0.2cm \evensidemargin -0.2cm 
\textwidth 16.5cm
\textheight 22.5cm \topmargin -0.46cm


\begin{document}


\begin{center}
{\large UNIVERSIDADE FEDERAL DO ESP\'{I}RITO SANTO} \\[0pt]
{\large Centro de Ci\^{e}ncias , Naturais e da Sa\'{u}de} \\[0pt]
\vspace*{0.5cm} {\large 1}$^{o}${\large \ Trabalho de F\'{\i}sica Experimental II - 2021/1}

Data: aplica\c{c}\~{a}o em 18/06/2021 e entrega at\'{e} 13/07/2021
\end{center}

Nome: 

\begin{enumerate}
\item Calcular (mostrando os c\'{a}lculos parciais) com 
n\'{u}mero de algarismos significativos corretos e dar a resposta em 
nota\c{c}\~{a}o cient\'{i}fica :

a) $(9,84\times 4,5\times 2097)/5,26$\hspace{7.5cm}

b) $1,9016m-9\times 10^{2}mm$

c) $5,46\times 10^{4}-5,9890\times 10^{2}$

\item Calcule numericamente (mostrando os c\'{a}lculos 
parciais), com incerteza :

a) $B-A-C$

b) $A\times B / C$

c) $C\times A^{2}/B$

considerando : \textit{A}= $5,24\pm 0,15$, \textit{B}= $8,203\pm 0,025$ e 
\textit{C}= $2,70\pm 0,04$. Compare as incertezas relativas de cada c\'{a}%
lculo com as de $A$, $B$ e $C$.

\item Um grupo de alunos realizou uma experi\^{e}ncia para verificar a conserva\c{c}\~{a}o de energia mec\^{a}nica. Eles colocaram um objeto de massa $M = (44,90 \pm 0,15)$g a uma certa altura de uma mesa e o abandonaram a partir do repouso em  uma rampa com altura $H (80 \pm 1)$ cm. No final da rampa eles mediram a velocidade do objeto que era de $V = (4,1 \pm 0,7)$ m/s. Calcule a varia\c{c}\~{a}o de energia potencial gravitacional e varia\c{c}\~{a}o de energia cin\'{e}tica. Elas s\~{a}o iguais? Justifique. Considere que acelera\c{c}\~{a}o da gravidade local seja $g = (9,786 \pm 0,005) m/s^{2}$.

\item  Um grupo de alunos realizou uma experi\^{e}ncia para determinar a resist\^{e}ncia el\'{e}trica ($R=V/I$) de um certo
resistor. Foi variada e medida a tens\~{a}o el\'{e}trica $V$ e medida a corrente el\'{e}trica $I$, tal que foi obtida a tabela abaixo :
\begin{table}[h]
\centering
\begin{tabular}{|c|c|c|}
\hline
N. & Tens\~{a}o (V) & Corrente I (mA) \\ \hline
1 & $1,2\pm 0,6$ & $4,3\pm 0,4$ \\ \hline
2 & $3,1\pm 0,6$ & $12,4\pm 0,6$ \\ \hline
3 & $5,6\pm 0,6$ & $20,2\pm 0,8$ \\ \hline
4 & $9,3\pm 0,6$ & $36,3\pm 1,0$ \\ \hline
\end{tabular}
\end{table}

a) Construa o gr\'{a}fico da tens\~{a}o el\'{e}trica $V$ versus corrente el\'{e}trica $I$, 
representando os va\-lo\-res com in\-cer\-te\-zas e tra\c{c}ando a reta m\'{e}dia.

b) Obtenha a resist\^{e}ncia el\'{e}trica ($R=V/I$) atrav\'{e}s do gr\'{a}fico com
incerteza, via regress\~{a}o linear.

c) A resist\^{e}ncia el\'{e}trica $R$ \'{e} compat\'{i}vel com um qual resistor (valor e cores) ?

\end{enumerate}

\begin{center}
\bigskip
\bigskip 

Prof. Roberto Colistete J\'{u}nior
\end{center}

\end{document}
